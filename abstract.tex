\subsubsection{内容简介}
本书的目标读者是准备去北美找工作的码农,也适用于在国内找工作的码农,以及刚接触ACM算法竞赛的新手。

本书包含了 HackerRank(\myurl{https://www.hackerrank.com/})所有题目的答案,
所有代码经过精心编写,编码规范良好,适合读者反复揣摩,模仿,甚至在纸上默写。

"ALGORITHMS"和"MISCELLANEOUS CODE GOLF"两部分的代码,使用C++ 11;"ARTIFICIAL INTELLIGENCE"和"FUNCTIONAL PROGRAMMING"两部分的代码,使用Scala。本书中的代码规范,跟在公司中的工程规范略有不同,为了使代码短(方便迅速实现):

\begindot
\item Shorter is better。能递归则一定不用栈;能用STL则一定不自己实现。

\item 不提倡防御式编程。不需要检查malloc()/new 返回的指针是否为nullptr;不需要检查内部函数入口参数的有效性。
\myenddot

本手册假定读者已经学过《数据结构》\footnote{《数据结构》,严蔚敏等著,清华大学出版社,
\myurl{http://book.douban.com/subject/2024655/}},
《算法》\footnote{《Algorithms》,Robert Sedgewick, Addison-Wesley Professional, \myurl{http://book.douban.com/subject/4854123/}}
这两门课,熟练掌握C++和Scala。

\subsubsection{GitHub地址}
本书是开源的,GitHub地址:\myurl{https://github.com/soulmachine/hackerrank}

\subsubsection{北美求职微博群}
我和我的小伙伴们在这里:\myurl{http://q.weibo.com/1312378}
